% Metódy inžinierskej práce

\documentclass[10pt,twoside,slovak,a4paper]{article}

\usepackage[slovak]{babel}
\usepackage[IL2]{fontenc} 
\usepackage[utf8]{inputenc}
\usepackage{graphicx}
\usepackage{url} % príkaz \url na formátovanie URL
\usepackage{hyperref} % odkazy v texte budú aktívne (pri niektorých triedach dokumentov spôsobuje posun textu)

\usepackage{cite}
%\usepackage{times}

\pagestyle{headings}

\title{Google Classroom,  MS Teams a ich využitie vo výučbe.\thanks{Semestrálny projekt v predmete Metódy inžinierskej práce, ak. rok 2020/21, vedenie: Meno Priezvisko}} % meno a priezvisko vyučujúceho na cvičeniach

\author{Filip Cák\\[2pt]
	{\small Slovenská technická univerzita v Bratislave}\\
	{\small Fakulta informatiky a informačných technológií}\\
	{\small \texttt{xcak@stuba.sk}}
	}

\date{\small 30. september 2020} % upravte



\begin{document}

\maketitle

\begin{abstract}
V mojom projekte sa plánujem zamerať na výhody a efektivitu využitia online platforiem Google Classroom a MS Teams pri výučbe a zároveň porovnať tieto dve služby. Počas pandémie boli školy nútené nájsť alternatívu prezenčného vzdelávania. Mnohé si vybrali práve jednu z týchto 2 alternatív.[1] V mojom projekte by som chcel bližšie opísať možnosti online vzdelávania cez Google Classroom alebo MS Teams[2], ktoré mohli byť zavedené aj skôr ako výpomoc pri komunikácii medzi žiakmi a učiteľmi. Zámerom je vyhodnotiť ako by to obom stranám uľahčilo komunikáciu, prehľadnosť zadaných úloh a cvičení ako aj následnú kontrolu. Poukázať aj na možný priestor zlepšenia spolupráce na tímových projektoch.
\end{abstract}



\section{Úvod}
%Motivujte čitateľa a vysvetlite, o čom píšete. Úvod sa väčšinou nedelí na časti.

Pandémia spôsobená vírusom COVID-19 sa začala ešte na jar 2020 a zasiahla mnohé časti spoločnosti, okrem iného aj školstvo. Školy stredné aj vysoké boli nútené zrušiť prezenčné vyučovanie a museli nájsť náhradu. Mnoho škôl na Slovnesku reagovalo promptne a začali vyučovať dištančne v online forme. Postupne sa pridávali ďalšie školy, ktoré skúšali rôzne platformy cez ktoré by mohli vyučovať. Napríklad Microsoft Teams, Google Classroom, Webex, Zoom. [3] V mojom projekte budem opisovať možné benefity Google Classrom a Microsoft Teams a ich využívnia aj počas prezenčného vyučovania.
 


Uveďte explicitne štruktúru článku. Tu je nejaký príklad.
Základný problém, ktorý bol naznačený v úvode, je podrobnejšie vysvetlený v časti~\ref{nejaka}.
Dôležité súvislosti sú uvedené v častiach~\ref{dolezita} a~\ref{dolezitejsia}.
Záverečné poznámky prináša časť~\ref{zaver}.
Základný problém, ktorý bol naznačený v úvode, je podrobnejšie vysvetlený v časti~\ref{nejaka}.
Dôležité súvislosti sú uvedené v častiach~\ref{dolezita} a~\ref{dolezitejsia}.
Záverečné poznámky prináša časť~\ref{zaver}.



\section{Google Classroom} \label{Google Classroom}

Google Classroom bol sprístupnený verejnosti už v roku 2014. [4] Táto e-learningova platforma najskôr využívala na lepšiu komunikáciu medzi učiteľmi a žiakmi a na vyššiu ekeftivitu - Google Docs na zápis poznámok, Drive na zdieľanie dokumentov a Gmail na komunikáciu. Postupne medzi tieto aplikácie pridali aj Google Sheets, Google Slides, Google Calendar, Google Jamboard a podobne. V roku 2020 integrovali aj službu Google Meet kde umožnujú učiteľom mať ku každej triede jedinečný link na videohovor. Google Classroom je pristupný cez webové prehliadače rovnako ako aj cez aplikácie na Chrome OS, iOS a Androide. Kedže ponúka všetky tieto možnosti, stáva sa ľahko prístupným a umožnuje hocikomu to používať. [3] Učitelia majú možnosť si tam ukladať všetky svoje práce rovnako ako aj žiaci. Do triedy je možné nahrať aj video ako z pohľadu učitela tak aj študenta. Študent má vo svojom rozhraní intuitívne vyobrazené miesta, kam odovzáva svoju prácu. Tú mu vie učiteľ jednoducho opraviť. V nasledujúcom odseku bližšie opíšem niektoré benefity.

\subsection{Google Classroom výhody} \label{Google Classroom: Google Classroom výhody}

\begin{itemize}
	\item Admin - može hromadne vytvoriť mailové adresy pre študentov
	\item Automaticky zmenené názvy odovzdaných súborov podľa mien študentov
	\item Jednoduchý prístup cez aplikáciu alebo prehliadač
	\item Možnost viacerých žiakov spolupracovať v jednom súbore.
	\item Zadarmo -
Google Classroom je momentálne zadarmo pre školy a aj pre žiakov. Nevyžaduje sa mať založený mail v Gmaili. Ak by chcel človek využívať nástroje ako Docs, Sheets, a podobne, no zároveň nie je šudentom školy ktorá využíva Google Classroom, je potrebné mať založený účet v Gmaili. 
	\end{itemize}




\section{Iná časť} \label{ina}

Základným problémom je teda\ldots{} Najprv sa pozrieme na nejaké vysvetlenie (časť~\ref{ina:nejake}), a potom na ešte nejaké (časť~\ref{ina:nejake}).\footnote{Niekedy môžete potrebovať aj poznámku pod čiarou.}

Môže sa zdať, že problém vlastne nejestvuje\cite{Coplien:MPD}, ale bolo dokázané, že to tak nie je~\cite{Czarnecki:Staged, Czarnecki:Progress}. Napriek tomu, aj dnes na webe narazíme na všelijaké pochybné názory\cite{PLP-Framework}. Dôležité veci možno \emph{zdôrazniť kurzívou}.


\subsection{Nejaké vysvetlenie} \label{ina:nejake}

Niekedy treba uviesť zoznam:

\begin{itemize}
\item jedna vec
\item druhá vec
	\begin{itemize}
	\item x
	\item y
	\end{itemize}
\end{itemize}

Ten istý zoznam, len číslovaný:

\begin{enumerate}
\item jedna vec
\item druhá vec
	\begin{enumerate}
	\item x
	\item y
	\end{enumerate}
\end{enumerate}


\subsection{Ešte nejaké vysvetlenie} \label{ina:este}

\paragraph{Veľmi dôležitá poznámka.}
Niekedy je potrebné nadpisom označiť odsek. Text pokračuje hneď za nadpisom.



\section{Dôležitá časť} \label{dolezita}



\section{Ešte dôležitejšia časť} \label{dolezitejsia}
\paragraph
zatial poznamky - odkazy aby som nezabudol 
[3]COVID-19 Pandemic: Langkawi Vocational College Student
Challenge in Using Google Classroom for Teaching and
Learning 

[4]  - debut google classroom
https://techcrunch.com/2014/05/06/google-debuts-classroom-an-education-platform-for-teacher-student-communication/


\section{Záver} \label{zaver} % prípadne iný variant názvu



%\acknowledgement{Ak niekomu chcete poďakovať\ldots}


% týmto sa generuje zoznam literatúry z obsahu súboru literatura.bib podľa toho, na čo sa v článku odkazujete
\bibliography{literatura}
\bibliographystyle{plain} % prípadne alpha, abbrv alebo hociktorý iný
\end{document}
